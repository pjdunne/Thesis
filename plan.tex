\documentclass[12pt]{article}
\usepackage{graphicx}

\usepackage[margin=1.0in]{geometry}

\begin{document}

\title{Thesis Plan}
\author{Patrick Dunne}

\maketitle
The focus of my thesis will be the search for invisible decays of Higgs bosons (Higgs to invisible) using the CMS detector at the LHC. I am funded until the end of March 2016 and intend to submit by then. The majority of the thesis will be about Higgs to invisible searches in the VBF production channel using both the prompt and parked CMS data from run I of the LHC. These searches have been completed.

One item that is ongoing is the interpretation of the results of CMS's Higgs to invisible searches. As a minimum I hope to update the limit Higgs to invisible results set on the spin independent dark matter nucleon interaction cross-section as a function of dark matter mass. As time allows I will also look into further theoretical interpretations with a view to interpreting results from run II of the LHC. I will also work on the VBF Higgs to invisible analysis for run II, although the scale of any contribution this makes to my thesis will depend on the timescale of the analysis. The total length of the thesis will be 120-150 pages and the chapter layout will be as follows.

\section{Theory $\sim$ 10 pages}
\begin{itemize}
\label{sec:theory}
\item Describe electroweak symmetry breaking in the Standard Model
\item Introduce dark matter and motivate search for it and other invisible final states in Higgs decays
\item Detail Higgs production and decays in order to motivate searching for invisible decays in the VBF channel
\end{itemize}

\section{Detector $\sim$ 15 pages}
\begin{itemize}
\label{sec:detector}
\item The LHC: brief overview of accelerator chain
\item The CMS detector: introduce trigger (including parked and prompt) and subsystems
\end{itemize}

\section{Physics objects $\sim$ 15 pages}
\begin{itemize}
\label{sec:physobjects}
\item Introduce all objects used in analysis:
\item[-] Jets, MET, electrons, muons, taus and photons if used in run II analysis
\end{itemize}

\section{Limit setting theory $\sim$ 5 pages}
\begin{itemize}
\label{sec:limits}
\item Go through CLs etc.
\end{itemize}

\section{Prompt data analysis $\sim$ 25-30 pages}
\begin{itemize}
\label{sec:promptana}
\item Introduction to selection and challenges of jets+met, e.g. trigger, QCD
\item Background estimation
\item Systematics
\item Results - focus on limit setting
\end{itemize}

\section{Parked data analysis $\sim$ 30-40 pages}
\begin{itemize}
\label{sec:parkedana}
\item As above for prompt data analysis but focus on differences and limit setting:
\item[-] New trigger efficiency characterisation, systematic improvements, signal region optimisation, differences in background estimations
\end{itemize}

\section{Combination and Dark matter interpretations $\sim$ 5-10 pages}
\begin{itemize}
\label{sec:interpretations}
\item Refer back to theory to introduce ZH and ggH searches
\item Present combination
\item At minimum present an update of spin independent cross-section limit against dark matter mass plot from prompt data EPJC paper
\item Present progress on interpretations that is made between now and writing up
\end{itemize}

\subsection{Towards/into run II $\sim$ 5-10 pages depending on progress}
\label{sec:run2}
\begin{itemize}
\item As above for prompt and parked data analyses depending on progress
\end{itemize}



\end{document}
