\chapter{Physics objects and event reconstruction}
\label{chap:obj}
%??Introduce reconstruction
This chapter describes the reconstruction of physics objects from data collected by the CMS detector. The invisible Higgs analysis uses a wide range of objects from the jets and \MET that are present in the signal process, to charged leptons that are present in background processes. This range of objects means that information from all subdetectors of CMS must be used. The reconstruction of each object used is described, along with the overarching ``particle flow'' approach to data reconstruction used by CMS.
%??Jets, MET, electrons, muons, taus 


\section{Primary vertex}
\label{sec:PV}
%??Describe PV reconstruction
%??Relevance to analysis
%??Check if hard scatter defined elsewhere
The very high instantaneous luminosities present at the \LHC lead to a large probability of multiple proton-proton interactons occuring in each bunch crossing. It is therefore essential to identify the \ac{PV}, which relates to the highest energy interaction or ``hard scatter''. Once this vertex has been identified it can be used to separate particles from the hard scatter and those from pileup. %!!secondary vertices

%??Algorithm
%??Performance


\section{Jets}
\label{sec:jets}
%??Describe jet reconstruction
%??Relevance to analysis
%??Algorithm
%??Performance


\section{Missing transverse energy}
\label{sec:MET}
%??Describe MET reconstruction
%??Relevance to analysis
%??Algorithm
%??Performance

\section{Electrons}
\label{sec:electrons}
%??Describe electron reconstruction
%??Relevance to analysis
%??Algorithm
%??Performance

\section{Muons}
\label{sec:muons}
%??Describe muon reconstruction
%??Relevance to analysis
%??Algorithm
%??Performance

\section{Taus}
\label{sec:taus}
%??Describe tau reconstruction
%??Relevance to analysis
%??Algorithm
%??Performance
