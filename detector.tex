\chapter{The \LHC and the \CMS experiment}
\label{chap:detector}
This chapter introduces the \CMS experiment and the \LHC~\cite{1748-0221-3-08-S08001}. In \SectionRef{sec:lhc} an overview of the \LHC and the chain of accelerators which feed into it is given. This is then followed in \SectionRef{sec:CMSInDetail} by a description of the \CMS experiment focusing on the aspects most relevant to the search for invisibly decaying Higgs bosons.

\section{The \LHC}
\label{sec:lhc}
The \LHC is situated 100\,m underground in a tunnel formerly built for the LEP accelerator~\cite{lepdesign} at CERN near Geneva, Switzerland. It is a 27\,km storage ring which accelerates both protons and heavy ions and collides them at the highest centre-of-mass energies of any collider built to date. The work contained in this thesis uses data from proton-proton collisions. These protons are obtained by taking hydrogen gas and stripping its atoms of their electrons with an electric field. The first accelerator in the chain of accelerators feeding into the \LHC, Linac 2, accelerates the protons to 50\,\MeV. The protons are then accelerated to 1.4\,\GeV~by the next accelerator, the \ac{PSB}, which is followed by the \ac{PS} where they reach 25\,\GeV. The beam energy is then increased to 450\,\GeV~in the \ac{SPS}. Finally, the protons are injected into the \LHC where, at the time of writing, the maximum energy the beams have been accelerated to is 6.5\,\TeV, close to the design maximum of 7\,\TeV.

When filled the \LHC contains two counter-rotating beams which are formed of up to 2808 bunches spaced either 25 ns or 50 ns apart and each containing $\mathcal{O}(10^{11})$ protons. The two beams are kept travelling in a closed orbit by 1232 superconducting dipole magnets and steered to four collision points around the \LHC. Detectors are situated at these collision points to observe the interactions, the main four being: ALICE~\cite{Aamodt:2008zz}, ATLAS, CMS and LHCb~\cite{Alves:2008zz}. A schematic of the chain of accelerators feeding into the \LHC and the \LHC detectors can be seen in \FigureRef{fig:lhclayout}.

\begin{figure}
  \includegraphics[width=\largefigwidth]{plots/detector/lhc_layout_sch.jpg}
  \caption[The layout of the chain of accelerators feeding into the \LHC, showing the position of the four main detectors.]{The layout of the chain of accelerators feeding into the \LHC, showing the position of the four main detectors~\cite{lhcexpschematic}.}
  \label{fig:lhclayout}
\end{figure}

When studying a physical process occurring in particle collisions it is important to know how many times it will occur, this can be expressed as:
\begin{equation}
  N = \mathcal{L}\sigma,
\end{equation}
where $\mathcal{L}$, the integrated luminosity, depends only on the parameters of the collisions, and the cross-section, $\sigma$, depends only on the process. In order to observe rare (i.e. low cross-section) processes, such as those studied at the LHC, it is necessary to use very high luminosity datasets. The integrated luminosity is obtained by integrating the instantaneous luminosity over time, so large luminosities can be obtained either by running the accelerator for a long time, or by operating at high instantaneous luminosity. For collisions at the \LHC the instantaneous luminosity is given by:
\begin{equation}
  \mathcal{L}_{inst}=\frac{k_{b}N_{b}^{2}f_{rev}\gamma}{4\pi\epsilon_{n}\beta}\,\cite{Benedikt:823808},
\end{equation}
where $k_{b}$ is the number of bunches per beam, $N_{b}$ the number of protons per bunch, $f_{rev}$ the revolution frequency, $\epsilon_{n}$ the normalised transverse beam emittance, $\beta^{*}$ the beta-function at the interaction point and $\gamma$ the Lorentz factor. The design instantaneous luminosity of the \LHC is $10^{34}\,\cm^{-2}\rm{s}^{-1}$ with 25 ns bunch spacing. The integrated luminosity is defined as $\mathcal{L}=\int\mathcal{L}_{inst}\mathrm{d}t$.

The \LHC started physics runs in 2010, during which it operated at a centre-of-mass energy of 7\,\TeV~ and delivered an integrated luminosity of 44.2 \invpb ~to CMS. In 2011 the \LHC also operated at 7\,\TeV~ and delivered 6.1\,\invfb to CMS. The centre-of-mass energy was increased to 8\,\TeV~ in 2012 and 23.3\,\invfb~of data were delivered to CMS. A summary of the luminosity delivered to CMS during the three periods of Run 1 can be seen in \FigureRef{fig:lumisummary}. In Run 2 the centre-of-mass energy was further increased to 13\,\TeV~and during 2015 4.09\,\invfb~of data were delivered to CMS at this energy. In order to be used for physics analysis data must be certified. This certification ensures that the detector was fully operational when the data were recorded. In 2011 5.1\,\invfb~were certified, in 2012 19.7\,\invfb~were certified and in 2015 2.2\,\invfb~were certified.


\begin{figure}
  \includegraphics[width=1.2\largefigwidth]{plots/detector/int_lumi_cumulative_pp_2.pdf}
  \caption[A summary of the luminosity delivered to CMS during Run 1 of the \LHC.]{A summary of the luminosity delivered to CMS during Run 1 of the \LHC \cite{CMSLumiPublic}.}
  \label{fig:lumisummary}
\end{figure}
%??maybe figure for 13 TeV too

The cross-sections for several processes are shown in \FigureRef{fig:xssummary} and it can be seen that the cross-section for VBF Higgs production is approximately 1.5 pb. Therefore, we expect approximately 30000 VBF-produced Higgs bosons in the 2012 dataset. By contrast the vector boson production cross-section is approximately 100 nb and the total cross-section for any process is orders of magnitude higher still. The separation of the relatively small number of signal events from the large background is a major challenge for the search for invisibly decaying Higgs bosons.

\begin{figure}
  \includegraphics[width=1.2\largefigwidth]{plots/detector/crosssections2012HE_v4.pdf}
  \caption[Cross-sections for several processes in collisions of protons with protons or anti-protons as a function of centre-of-mass energy. The energies that the \LHC and Tevatron ran at are highlighted.]{Cross-sections for several processes in collisions of protons with protons or anti-protons as a function of centre-of-mass energy. The energies that the \LHC and Tevatron ran at are highlighted~\cite{Stirlingppxs}.}
  \label{fig:xssummary}
\end{figure}

The large total cross-section combined with the high instantaneous luminosities that the \LHC operates at leads to the probability for multiple proton-proton interactions per bunch crossing being high. The distribution of the number of interactions per bunch crossing, $\mu$, can be seen in \FigureRef{fig:pusummary}. The additional interactions on top of the process of interest in a bunch crossing are called \ac{PU}.

\begin{figure}
  \includegraphics[width=1.2\largefigwidth]{plots/detector/pileup_pp_2012.pdf}
  \caption[Distribution of the number of interactions per bunch crossing in CMS during 2012 running of the \LHC.]{Distribution of the number of interactions per bunch crossing in CMS during 2012 running of the \LHC \cite{CMSLumiPublic}.}
  \label{fig:pusummary}
\end{figure}


\section{The \CMS experiment}
\label{sec:CMSInDetail}
%Introduction - describe hermetic onion shell concept and introduce subsystems
The CMS detector was designed to search for the SM Higgs and new physics at the TeV energy scale. Both because the nature of new physics is not known and because the SM Higgs has a wide range of decays and production mechanisms CMS must be sensitive to many different types of final state particles and topologies. In order to achieve this it has a hermetic design comprising a barrel, endcaps and a forward calorimetry system. It is composed of several layers of subdetectors each sensitive to different particles as shown in \FigureRef{fig:cmsschematic1}. The hermiticity of the detector is particularly important for the VBF Higgs to invisible search. Further details on the CMS detector beyond those in this section can be found in \ReferenceRef{Chatrchyan:2008aa}. %because, as described in \SectionRef{sec:higprod}, the VBF final state is highly likely to have jets in the forward regions of the detector.

\begin{figure}
  \includegraphics[width=1.2\largefigwidth]{plots/detector/cms_120918_03.png}
  \caption[A diagram of the subsystems making up the CMS detector, illustrating the hermeticity and layered structure of the experiment.]{A diagram of the subsystems making up the CMS detector, illustrating the hermeticity and layered structure of the experiment~\cite{cmsschematic}.}
  \label{fig:cmsschematic1}
\end{figure}

A central design feature of CMS is the superconducting magnet, inside which is generated a 3.8\,T axial field. This field bends the path of charged particles travelling through it allowing their momentum to be measured. Not all particles are charged however, and the path of several types of particles through the CMS detector is shown in \FigureRef{fig:cmsschematic2}. The first layer is the tracker which records the paths taken by charged particles. As well as providing a momentum measurement the tracks also allow the vertex from which the particle came to be identified. The next layer is the \ac{ECAL} where electrons and photons deposit energy through electromagnetic showers. This is followed by the \ac{HCAL} where hadrons deposit most of their energy. After the calorimetry systems is the superconducting magnet which is not instrumented. Outside the magnet are the muon detection systems, which are interspersed with iron plates which form the return yoke for the magnet. Due to their high mass compared to electrons, muons do not deposit much energy in the detector and often are not stopped, so the muon system is primarily a tracking detector.

\begin{figure}
  \includegraphics[width=1.2\largefigwidth]{plots/detector/CMS_Slice.png}
  \caption[A schematic cross-section of the CMS experiment showing the path taken by several types of particles.]{A schematic cross-section of the CMS experiment showing the path taken by several types of particles~\cite{CMSSlice}.}
  \label{fig:cmsschematic2}
\end{figure}

%Coordinate system
The origin of the co-ordinate system used by CMS is at the nominal interaction point. It is a right handed cartesian system with the $x$ axis pointing towards the centre of the \LHC ring and the $y$-axis vertically upwards, the $z$ axis then points along the beam line. The azimuthal angle $\phi$ and the polar angle $\theta$ are measured in radians from the $x$ and $z$ axes respectively. It is common to describe the direction of outgoing particles using $\phi$ and their pseudo-rapidity, $\eta$ which is defined as:
\begin{equation}
  \label{eq:eta}
  \eta=-\ln[\tan(\theta/2)].
\end{equation}
Distances in the $\eta-\phi$ plane are given by $\Delta R=\sqrt{\Delta\phi^2+\Delta\eta^2}$. Two other quantities often used at hadron colliders are the projections of a particle's momentum and energy in the transverse plane, these are denoted as \pt and \Et respectively. The missing transverse energy, defined as the negative vector sum of the momentum in the transverse plane of all particles in an event, is important in inferring the presence of invisible particles and is denoted \MET. Also, when describing an event the terms ``leading'' and ``sub-leading'' refer to the highest and second-highest \pt objects in an event respectively.

\subsection{Tracker}
\label{sec:tracker}
%Describe Tracker http://dx.doi.org/10.1140/epjcd/s2004-03-1801-1 and jinst paper
The tracker is designed to measure the paths of charged particles precisely from \LHC collisions which curve in CMS's magnetic field. The design transverse momentum resolution of the full tracking detector is 1-2\% at 100\GeV~. In order to measure the particles' positions precisely and ensure the occupancy of the tracker is low a high granularity is required. Due to the frequency of collisions at the \LHC and the high instantaneous luminosity a radiation hard system with fast response is also necessary. This combination of requirements motivates the use of a silicon-based system. When traversing silicon charged particles create electron-hole pairs, which are then separated by an applied electric field, causing a current pulse.

%Submodules
%pixel
The tracker layout can be seen in \FigureRef{fig:trackerschematic}. In order to keep the sensor occupancy below 1\% at design luminosity, the innermost component is a silicon pixel detector. This detector has three layers in the barrel, at radii of 4.7, 7.3 and 10.2 cm, and two in the endcap. The are 66 million pixels each 100\,\micron\,$\times$ 150\,\micron\,in size. The resulting resolution of the pixel detector is approximately 10\,\micron\,in the $r-\phi$ plane and 17\,\micron\,in the $r-z$ plane \cite{trackerperformance}. During Run 1 the proportion of modules in the pixel (strip) tracker known to be defective was 2.4\% (2.3\%)~\cite{Chatrchyan:1704291}.

\begin{figure}
  \includegraphics[width=1.2\largefigwidth]{plots/detector/TrackerSchematic.png}
  \caption[A cross-section of the CMS tracker, indicating the subsystems that comprise it. Each line indicates a detector module and the labels for each subsection are the names given by CMS to the various subsections of the tracker.]{A cross-section of the CMS tracker, indicating the subsystems that comprise it. Each line indicates a detector module~\cite{Chatrchyan:2008aa} and the labels for each subsection are the names given by CMS to the various subsections of the tracker.}
  \label{fig:trackerschematic}
\end{figure}



%strip
Surrounding the pixel detector is a silicon strip detector with 10 layers in the barrel, at radii of 20 to 116 cm, and 12 pairs of disks in the endcap. The strips are typically 10-20 cm long and 80-180\,\micron\,wide, with the strip size increasing with radius as the particle flux decreases. The strip detector's single point resolution is 230-530\,\micron\,in the $r-z$ plane and 23-52\,\micron\,in the $r-\phi$ plane.  The better resolution in the $r-\phi$ plane allows \pt to be measured with higher precision, as this is the direction in which a particle's track bends in the CMS magnetic field. The barrel and endcap detectors together have an acceptance of $|\eta|<2.5$ for both the pixel and strip detectors. Further details on the position resolution of the tracking detector for vertex reconstruction will be given in \SectionRef{sec:PV}.

\subsection{Electromagnetic calorimeter}
\label{sec:ECAL}
%Describe \ac{ECAL}
The \ac{ECAL} is designed to provide accurate photon and electron reconstruction and precise measurement of the electromagnetic component of hadron jets. It is a homogeneous calorimeter made of lead tungstate (PbWO$_{4}$) crystals, separated into a barrel section, with 61200 crystals and two endcaps each with 7234 crystals. These crystals are 25.8 radiation lengths in depth in the barrel and instrumented with photodetectors (avalanche photodiodes in the barrel and vacuum phototriodes in the endcap). 

The layout of the \ac{ECAL} is shown in \FigureRef{fig:ecalschematic}. The \ac{EB} crystals have a 170$\times$360 arrangement in $\eta-\phi$ space such that the gaps between crystals are offset by $3^{\circ}$ from the vector to the detector origin, thus avoiding particles travelling through the gaps. The \ac{EB} extends to $|\eta|=1.479$, with higher values of $\eta$ covered by the \ac{EE}. The crystals in the \ac{EE} are arranged in an $x-y$ grid pointing at a focus 1.3\,m from the nominal interaction point, giving a $2^{\circ}-8^{\circ}$ separation between the gaps between crystals and the vector to the detector origin. In addition to the main PbWO$_{4}$ detector the endcaps also have a preshower detector. This preshower is a lead silicon strip sampling calorimeter, which initiates the electromagnetic showers and provides sufficient position resolution to distinguish single photons from pairs produced in neutral pion decays. The total acceptance of the barrel and endcap detectors is $|\eta|<3.0$.

\begin{figure}
  \includegraphics[width=1.2\largefigwidth]{plots/detector/ecal_layout.png}
  \caption[A schematic of the CMS ECAL, indicating the subsystems that comprise it. The ECAL is 7.8\,m long by 3.5\,m wide.]{A schematic of the CMS ECAL, indicating the subsystems that comprise it. The ECAL is 7.8\,m long by 3.5\,m wide~\cite{Chatrchyan:2008aa}.}
  \label{fig:ecalschematic}
\end{figure}

 On entering the \ac{ECAL} high energy electrons or photons initiate an electromagnetic shower by undergoing Bremsstrahlung or pair production respectively. The resulting cascade of particles continues to lose energy by successive Bremsstrahlung and pair production until their energy is low enough that the photons no longer undergo pair-production and the electrons lose their energy mainly by ionisation. The excitation of the PbWO$_{4}$ crystals leads to the emission of scintillation light, proportional to the amount of energy deposited, which is collected by the photodetectors.

The choice of PbWO$_{4}$ is motivated by its high density (8.28 g/cm$^{3}$), short radiation length (0.89 cm), small Moli\'{e}re radius (2.2\,cm) and radiation hardness. These properties lead to the showers being contained in a small area and allows the calorimeter to be compact and have fine granularity. Another advantage of PbWO$_{4}$ is that 80\% of the scintillation light is emitted within the LHC's 25\,ns design bunch-crossing time, so particles can be properly associated with the bunch-crossing from which they originate.

%Resolution
For particle energies below 500 \GeV, for which the resulting showers are to be contained in the full depth of the \ac{ECAL}, the \ac{ECAL} resolution can be parameterised as:
\begin{equation}
  \label{eq:ecalres}
  \left(\frac{\sigma}{E}\right)^2=\left(\frac{S}{\sqrt{E}}\right)^2+\left(\frac{N}{E}\right)^2+C^2.
\end{equation}
Where $S$ is the stochastic term, $N$ the noise term and $C$ the constant term. The stochastic term is comprised of fluctuations in the lateral containment of showers and also in the amount of scintillation light. The noise term is made up of electronic and digital noise, and signals from other bunch crossings which do not fully dissipate in time. The constant term comes from non-uniformity of light collection along the crystals, errors in the calibration of crystals against each other and leakage of energy from the back of the calorimeter. The energy resolution was measured without an applied magnetic field in an electron beam using particles with momenta between 20 and 250\,\GeV. The stochastic, noise and constant terms were found to be 0.028 $\GeV^{1/2}$, 0.12\,\GeV and 0.003 respectively.

As the \ac{ECAL} is exposed to radiation the PbWO$_{4}$ crystals darken and as a result fewer photons are collected per unit energy deposited. The loss of response due to this darkening at the end of Run I varies from 6\% for crystals in the most central region of the \ac{ECAL} to 30\% in the endcaps~\cite{CMS-DP-2015-063}.

 

\subsection{Hadronic calorimeter}
\label{sec:HCAL}
%Describe HCAL
The \ac{HCAL} is designed to measure the energy of strongly interacting particles. This measurement is particularly important for neutral hadrons which do not leave tracks in the tracking system and deposit most of their energy in the \ac{HCAL}, and for the determination of \MET.  The main part of the \ac{HCAL} consists of a brass and scintillator plus wavelength shifting fibre sampling calorimeter split into \ac{HB} and \ac{HE} sections. The primary design consideration for the \ac{HCAL} is that it must fit between the outer edge of the \ac{ECAL} ($r=1.77$ m) and the inner edge of the magnet ($r=2.95$ m). In order to satisfy this requirement and achieve satisfactory containment of hadronic showers the magnet coil is also used as an absorber, and there is a further layer of scintillator outside the magnet coil (\ac{HO}). The barrel and endcap detectors extend to $|\eta|<3$.

Brass is chosen as the main \ac{HCAL} absorber because it is not magnetic and has a relatively short nuclear interaction length of 16.42 cm. Once showers have been initiated in the absorber layers they then pass through the plastic scintillator tiles, where they create pulses of light. These pulses are transferred via wavelength shifting fibres to hybrid photodiodes. The segmentation of the scintillator is such that the $\eta-\phi$ resolution in the \ac{HB} (\ac{HE}) is $0.087\times 0.087$ (between $0.087\times 0.087$ and $0.17\times 0.17$ depending on $\eta$).

In addition to the barrel and endcap sections of the \ac{HCAL} there is also a steel and quartz fibre Cherenkov forward calorimeter (\ac{HF}), which extends the calorimetry coverage of CMS to $|\eta|<5.2$. The choice of this technology is driven by its ability to withstand the very high particle fluxes present so close to the beamline. Showers are initiated by the steel absorber and signals are generated in the quartz fibres by particles above the Cherenkov threshold generating Cherenkov light, which is collected by photomultiplier tubes. Due to the Cherenkov energy threshold increasing with particle mass the \ac{HF} is primarily sensitive to the electromagnetic component of showers.

A diagram of the \ac{HCAL} layout can be seen in \FigureRef{fig:hcalschematic}. In total the \ac{HCAL} corresponds to 10-15 interaction lengths, depending on $\eta$. The resolution of the barrel and endcap sections of the \ac{HCAL} as a function of the incident particle energy was measured in a pion beam and has been found to be well parameterised by:
\begin{equation} 
  \label{eq:hcalres}
  \left(\frac{\sigma}{E}\right)^{2}=\left(\frac{94.3\%}{\sqrt{E}}\right)^{2}+\left(8.4\%\right)^{2}~\cite{Abdullin:2009zz}.
\end{equation}

\begin{figure}
  \includegraphics[width=1.2\largefigwidth]{plots/detector/hcal_layout1.png}
  \caption[A schematic of a quadrant of the CMS HCAL in the $r-z$ plane, indicating the subsystems that comprise it.]{A schematic of a quadrant of the CMS HCAL in the $r-z$ plane, indicating the subsystems that comprise it~\cite{Chatrchyan:2008aa}.}
  \label{fig:hcalschematic}
\end{figure}

\subsection{Muon system}
%Describe muon system
As described above muons are highly penetrating, and thus are only rarely contained by the inner detector. Very few other charged particles are able to leave the calorimeters without being absorbed, so the presence of tracks in the muon system is sufficient to identify muons. The muon tracking system uses three types of gaseous particle detectors, located throughout the iron magnet return yoke. In all three types of detector when a charged particle travels through the gaseous detector it ionises the gas, the resulting free electrons then drift towards the detector's anode resulting in an electrical signal. The two primary types of detectors used are the \ac{DT}, which is used in the barrel section of the detector ($|\eta|<1.2$), and the \ac{CSC}, which is used in the endcap ($0.9<|\eta|<2.4$). The \ac{DT} and \ac{CSC} systems identify muons and provide measurements of their momentum. These measurements can be combined with those from the tracker to improve the muon momentum resolution. This combined reconstruction and momentum measurement along with its resolution is described in \SectionRef{sec:muons}. Additionally there is a \ac{RPC} system in both the barrel and endcap regions ($|\eta|<1.6$), the primary purpose of which is to provide trigger and bunch-crossing identification information. A diagram of the CMS muon system can be found in \FigureRef{fig:muonschematic}.

%DT
Each system has its own particular advantages and disadvantages which make it best suited for use in the various parts of the muon system. \ac{DT}s are inexpensive and reliable, but they are not usable in regions with high muon and neutron background rates, making them well suited to the barrel portion of the detector, where large areas must be instrumented and rates are low. Each \ac{DT} is a 2.4\,m long wire in a $13\times 42$\,mm$^{2}$ tube. The length is limited by the segmentation of the iron return yoke, and the cross-section by the requirement that the occupancy and drift time are low enough to prevent multiple muon hits being read out at the same time. The \ac{DT}s are organised in 4 stations, interspersed with return yoke iron plates. The first three stations have 8 chambers each, 4 to measure the muon's position in the $r-\phi$ plane and 4 to measure the $z$ co-ordinate. The final outermost layer does not have the $z$-measuring chambers. These chambers consist of 8-12 stacked \ac{DT}s, with each layer offset from the previous one by half the width of a tube to avoid gaps.

%CSC
Due to their fast response time, fine segmentation and radiation resistance \ac{CSC}s are ideal for the endcap region where the muon and background rates are higher. Each \ac{CSC} is a multiwire proportional chamber with 7 planes of cathode strips running radially outwards with 6 planes of anode wires, which run azimuthally, interleaved between them. Both the anode and cathode wires are read out to provide $\eta$ and $r-\phi$ co-ordinate measurements respectively. Similarly to the \ac{DT} system the design number of \ac{CSC} stations in each endcap is 4 interspersed with iron return yoke plates. During Run 1 only three of the \ac{CSC} stations were present; the fourth station in each endcap was added during the long shutdown and is present for Run 2. The position resolution in the $r-\phi$ plane of the \ac{CSC}s varies from 75-80\,\micron.

%RPC
The \ac{RPC}s are gas gaps surrounded by anode and cathode plates with read out strips between them. The advantage of \ac{RPC}s is that their response is good at high rates, and they have very good time resolution, making them ideal for use in the trigger and assignment of muons to a bunch crossing. However, they have much poorer position resolution than the \ac{DT}s or \ac{CSC}s. There are 6 layers of \ac{RPC}s in the barrel and 3 in the endcap.

\begin{figure}
  \includegraphics[width=1.2\largefigwidth]{plots/detector/muon_layout.pdf}
  \caption[A schematic of a quadrant of the CMS muon system in the $r-z$ plane, indicating the subsystems that comprise it.]{A schematic of a quadrant of the CMS muon system in the $r-z$ plane, indicating the subsystems that comprise it~\cite{Bayatian:922757}.}
  \label{fig:muonschematic}
\end{figure}

\subsection{Trigger system}
\label{sec:triggers}
%Describe trigger system
The design bunch crossing rate of the \LHC is 40 MHz, and for the data used in this thesis it was either 20 or 40 MHz. Since each event consists of approximately 1 MB of data, writing every event to tape would correspond to a data rate of 20-40 TB$/s$ which is not feasible. It is also not feasible for the detector electronics to read out the detector at this frequency. It is therefore necessary to use a trigger system to perform an ``online'' reconstruction and reduce the event rate by selecting only the most interesting events.

The trigger is separated into two stages, the \ac{L1} trigger and the \ac{HLT}. First the \ac{L1} trigger, which is built of custom-designed electronics, reduces the rate to a maximum of 100 kHz. The decision to accept an event in the \ac{L1} trigger or not starts with local information on the energy deposits in the calorimeters and hits in the muon systems, which is stored for all events for 128 bunch crossings. A decision must therefore be made within 128 bunch crossings or the event is discarded. Due to the limited time available and the limited available bandwidth of the data acquisition system, this information is generally not available at the detector's full resolution. After it is collected from the detector the local information is then passed to the regional trigger systems, which generate lists of trigger candidates, such as electrons or jets, ranked by energy and quality. These ranked lists from each region are then passed to the global muon and calorimeter system triggers, which select the highest ranked candidates across the whole detector and give them to the global trigger, which makes a final decision. This process is shown in \FigureRef{fig:l1layout}.

\begin{figure}
  \includegraphics[width=1.2\largefigwidth]{plots/detector/L1T_Layout.png}
  \caption[A schematic of the L1 trigger system. The arrows indicate the flow of data, the information transferred between systems is also indicated.]{A schematic of the L1 trigger system. The arrows indicate the flow of data, the information transferred between systems is also indicated~\cite{Chatrchyan:2008aa}.}
  \label{fig:l1layout}
\end{figure}

If an event is accepted by the \ac{L1} trigger the full detector information is read out to the \ac{HLT} farm on the surface, which reduces the rate further still to approximately 1 kHz.  The \ac{HLT} consists of several thousand commercially available CPUs. Despite having the full detector information, the time available does not allow for the full offline reconstruction to be performed. Nevertheless, the algorithms available at the \ac{HLT} are much closer to those used offline than those available at \ac{L1}, allowing the trigger to better select events that would also pass requirements on the offline quantities. If they are accepted by the \ac{HLT}, events are sent to be reconstructed using the \ac{WLCG}. 

\subsection{Data processing}
The \ac{WLCG} consists of several tiers. Data is first fully reconstructed at the Tier 0 centres. During Run I there was only one of these at CERN, for Run II there will also be a Tier 0 centre in Budapest. It is then sent to at least one Tier 1 centre, so that a full copy of the data is available at multiple sites in different geographic locations. Tier 2 and 3 centres then process this data according to the needs of specific analyses.

During 2012 running it was realised that it was possible for data to be written to tape from the CMS detector at a higher rate than it could be reconstructed by the Tier 0. 30\% of the output of CMS was therefore immediately sent for ``prompt'' reconstruction, while the remainder was ``parked'' to tape to be reconstructed during \LHC shutdown periods when there is spare computing capacity available \cite{CMS-DP-2012-022}. The extra events that could be stored through this parking allowed significantly lower trigger selection thresholds to be used for some of the analyses described in this thesis.


