%% Title
%??UPDATE THIS
\titlepage[of Imperial College London]%
{A dissertation submitted to Imperial College London\\
  for the degree of Doctor of Philosophy}

\newpage
The copyright of this thesis rests with the author and is made available under a Creative Commons
Attribution Non-Commercial No Derivatives licence. Researchers are free to copy, distribute or
transmit the thesis on the condition that they attribute it, that they do not use it for commercial
purposes and that they do not alter, transform or build upon it. For any reuse or redistribution,
researchers must make clear to others the licence terms of this work

%% Abstract
\begin{abstract}%[\smaller \thetitle\\ \vspace*{1cm} \smaller {\theauthor}]
  %\thispagestyle{empty}
  Searches for invisibly decaying Higgs bosons using the Compact Muon Solenoid (CMS) detector at the CERN Large Hadron Collider (LHC) and their interpretations are described. These searches are motivated by the cosmological observation of very weakly interacting matter in the universe called dark matter. In order to provide context for these searches, introductions to the Standard Model (SM) of particle physics and several extensions to the SM which incorporate dark matter are given.

The searches described in this thesis use data recorded in proton-proton collisions in 2012 and focus on the most sensitive Vector Boson Fusion (VBF) production mode. The first search uses 19.5 \invfb\, of data promptly reconstructed in 2012 and results in an observed (expected) limit on the invisible branching fraction of the 125 \GeV Higgs boson, \BRinv, of 0.65 (0.49) at the 95\% confidence level (CL). The second search uses 19.2 \invfb\,of data which was collected using triggers with looser thresholds and reconstructed later, in 2013. This search resulted in an observed (expected) limit on \BRinv of 0.57 (0.40) at the 95\% CL.

Combinations of these searches with searches in other production channels are also described, the most sensitive of which results in an observed (expected) limit on \BRinv of 0.36 (0.30) at 95\% CL. Projections of the sensitivity of these analyses in Run 2 and interpretations of their results as limits on various models of dark matter are also given.
\end{abstract}


%% Declaration
\begin{declaration}
  This dissertation is the result of my own work. Where figures and results are taken from other sources this is indicated by an appropriate reference. Some figures referenced as from other sources were created by me, but appear in other public documents. 

The description of the analysis described in \ChapterRef{chap:prompt} follows \ReferenceRef{ARTICLE:CMSAN-12-403}, which includes elements of my work. Specifically, I was responsible for cross-checking the results of all the $\PW+$jets background estimations, I contributed to the development of the formula used to carry out the $\PZ+$jets background estimation, I calculated and cross-checked several of the systematic uncertainties, I produced plots of the discriminating variables for final publication, and I performed all limit setting and production of limit plots. This work was made public in \ReferenceRef{Chatrchyan:2014tja}.

I was the main analyser for the analysis described in \ChapterRef{chap:parked}, responsible for all steps of the analysis including trigger efficiency measurement, background estimation, systematic uncertainty studies and limit setting. The QCD background estimation method was developed and implemented collaboratively with other members of the Imperial College high energy physics (HEP) group. This work was made public in \ReferenceRef{CMS-PAS-HIG-14-038}.

I was also solely responsible for the combinations described in \ChapterRef{chap:comb}. For the interpretations of the VBF invisibly decaying Higgs boson searches as limits on dark mater models described in \ChapterRef{chap:interp} I worked collaboratively with theorists at Rutgers University and members of the Imperial College and Bristol University HEP groups. This work was made public in \ReferenceRef{ourdmpaper}.

As well as the work described in this thesis I have also carried out preparations for a search for VBF produced invisibly decaying Higgs bosons using Run 2 data from the LHC. These preparations have included measurements of the trigger efficiency for the triggers used in 2015 CMS data taking and comparisons of the kinematic distributions of signal and background events at the increased 13 TeV centre of mass energy used in Run 2 with those from Run 1. Together with a new PhD student, I have also carried out a combination of the first Run 2 invisibly decaying Higgs boson searches in the VBF and ZH channels with those in Run 1, which is currently being approved for publication.


  \vspace*{1cm}
  \begin{flushright}
    Patrick James Dunne
  \end{flushright}
\end{declaration}


%% Acknowledgements
\begin{acknowledgements}
%??Stephanie, Mum, Dad rest of family
%??Gavin, Dave, Anne Marie, Joao, Jim, Andrew, Nick all other colleagues
%??STFC, and IC
\end{acknowledgements}


%% Preface
%\begin{preface}
%\end{preface}

%% ToC
\tableofcontents

%% Strictly optional!
%\frontquote%
%{Writing in English is the most ingenious torture\\
%   ever devised for sins committed in previous lives.}%
%  {James Joyce}
