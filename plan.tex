\documentclass[12pt]{article}
\usepackage{graphicx}

\usepackage[margin=1.0in]{geometry}

\begin{document}

\title{Thesis Plan}
\author{Patrick Dunne}

\maketitle
Brief summary of work so far (say work was sufficient for publication and PAS), work still to do and funding and writeup timeline


One of the first tasks in my PhD was participating in the cross-check of a search for VBF produced Higgs decays to invisible final states (H$\rightarrow$inv.) using data from run I of the LHC collected with the promptly reconstructed triggers (prompt data). I also performed the combination of the prompt data VBF analysis with the two CMS H$\rightarrow$inv. analyses in the ZH production channel. These prompt data analyses were published in EPJC and will form the “Prompt” section of the thesis. In addition to the analysis of prompt data I also took part in a search for VBF produced H$\rightarrow$inv. using data collected in run I with parked triggers. I again combined this parked data analysis with the two CMS Higgs to invisible analyses in the ZH production channel. This work was made public as a CMS Physics Analysis Summary and will form the “Parked” section of the thesis.

In order to complete the thesis plan I need to do more work in the interpretations of the CMS Higgs to invisible searches. The minimum aim that I have is to update figure 13 from the Higgs to invisible EPJC paper to take into account the new parked data VBF analysis and theoretical considerations which render the fermion interpretation invalid. If time permits I will also investigate the interpretation of limits on Higgs decays to invisbile final states in other theoretical models, especially with a view to possible run 2 analyses.

In parallel with this I will continue to work on the run 2 Higgs to invisible analysis, although it is unclear how much of this will contribute to my thesis due to the LHC timeline this year.


The total length of the thesis will be 120-150 pages and the chapter layout will be as follows.

\section{Theory $\sim$ 10 pages}
\begin{itemize}
\label{sec:theory}
\item Describe electroweak symmetry breaking in the Standard Model
\item Higgs production and decays
\item Introduce dark matter
\item Motivate search for dark matter and other invisible final states in Higgs decays
\end{itemize}

\section{Detector $\sim$ 15 pages}
\begin{itemize}
\label{sec:detector}
\item The LHC: brief overview of accelerator chain
\item CMS: talk about parked vs. prompt here
\end{itemize}

\section{Physics objects $\sim$ 15 pages}
\begin{itemize}
\label{sec:physobjects}
\item Introduce all objects used in analysis:
\item[-] Jets, MET, electrons, muons, taus, photons if used in run II
\end{itemize}

\section{Limit setting theory $\sim$ 5 pages}
\begin{itemize}
\label{sec:limits}
\item Go through CLs etc.
\end{itemize}

\section{Prompt data analysis $\sim$ 25-30 pages}
\begin{itemize}
\label{sec:promptana}
\item Introduction to experimental challenges of jets+MET
\item Background estimation
\item Systematics
\item Results - focus on limit setting
\end{itemize}

\section{Parked data analysis $\sim$ 30-40 pages}
\begin{itemize}
\label{sec:parkedana}
\item New trigger efficiency characterisation, systematics, signal region optimisation, differences in background estimations
\item Results - focus on limit setting
\end{itemize}

\section{Combination and Dark matter interpretations $\sim$ 5-10 pages}
\begin{itemize}
\label{sec:interpretations}
\item Refer back to theory to introduce ZH searches
\item Present combination
\item At minimum present an update of spin independent cross-section limit against dark matter mass plot from prompt data EPJC paper
\item Present progress on interpretations that is made between now and writing up
\end{itemize}

\subsection{Towards/into run II $\sim$ 5-10 pages depending on progress}
\label{sec:run2}




\end{document}
