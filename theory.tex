\chapter{Introduction and theory}
\label{chap:theory}
This chapter will explain the theory of the Higgs boson. It will start with an introduction to the standard model (SM), focussing on the Higgs mechanism, before outlining the motivations behind and some candidates for physics beyond the SM (BSM). Natural units, where $\hbar=c=1$, Einstein summation convention and Feynman slash notation are used throughout. Four vector indices are labelled using greek letters, SU(2) generators are labelled using i,j and k, and SU(3) generators are labelled using a,b and c.

\section{The standard model of particle physics}
\label{sec:SM}
The SM describes the interaction of the particles currently thought to be fundamental with the strong, weak and electromagnetic forces. Its predictions, which come solely from specifying the particles in the theory, the symmetries the theory respects, how they are broken and 18 free parameters have been tested in many different experiments in some cases up to one part in a trillion \cite{PhysRevLett.100.120801}. However, it does face challenges, described in section \SectionRef{sec:SMchallenges}, one example being that it does not describe dark matter. 

The SM is a gauge invariant quantum field theory (QFT). In order to construct a QFT the symmetries that are respected by the theory and the fields it describes must be specified. Symmetries are important because of Noether's theorem, which states that for every continuously differentiable symmetry of the Lagrangian of a theory there is a corresponding conservation law~\cite{Noether:1918zz,doi:10.1080/00411457108231446}. An example of this is that we observe that the laws of physics are invariant under translations and rotations in space and time, this is known as Poincar\'e invariance. These simple requirements lead through Noether's theorem to the conservation of energy, linear momentum and angular momentum. In addition to giving rise to conservation laws, some types of symmetry lead to additional fields being required to preserve invariance \cite{PhysRev.96.191}.

The particles must be specified, because in a QFT they correspond to the quantised excitations of fields. Specifically, scalar fields correspond to spin zero bosons, spinor fields correspond to spin half fermions, and vector fields correspond to spin 1 bosons. Therefore, the fields that can be present are limited by the particles that are observed. In order to add a new field an explanation for why the corresponding particle has not yet been observed must, therefore, be provided.

\subsection{Particles and forces in nature}
\label{sec:symmetriesandfields}
As discussed in \SectionRef{sec:SM}, the properties of the universe place powerful constraints on the terms that enter the Lagrangian of any theory describing it. The observed particles in the universe fall into two types, fermions and bosons.

%!!Say what particles and forces in nature give representations of fermions

\subsection{Introduction to gauge symmetries}
\label{sec:gaugesym}
As the SM is a gauge invariant QFT gauge symmetries will now be introduced. Gauge symmetries are local transformations, i.e. the transformation can be different at different points in space and time, that form a symmetry group. To see the effect of imposing such a symmetry on a theory consider imposing local invariance under U(1) transformations on the Dirac Lagrangian for a masssive fermion:
\begin{equation}
  \label{eq:globaldiraclagrangian}
  \mathcal{L}=i\bar{\psi}\slashed{\partial}\psi-m\bar{\psi}\psi.
\end{equation}

This Lagrangian is invariant under a global $U(1)$ transformation $\psi\rightarrow e^{iq\theta}\psi$. However, if the $U(1)$ transformation is local i.e. $\theta$ is a function of spacetime position the Lagrangian is no longer invariant and transforms as:
\begin{equation}
  \label{eq:gaugeviolating}
  \mathcal{L}\rightarrow\mathcal{L}-q(\partial_{\mu}\theta)\bar{\psi}\gamma^{\mu}\psi.
\end{equation}

In order to restore invariance a vector field, $A_{\mu}$, referred to as a gauge field or gauge boson, which transforms as $A_{\mu}\rightarrow A_{\mu}+\partial_{\mu}\theta$ and has an interaction with the fermion field:
\begin{equation}
  \mathcal{L}_{int}=q(\bar{\psi}\gamma^{\mu}\psi) A_{\mu},
\end{equation}
 can be added to the theory. The interaction term of the new gauge field transforms as:
\begin{equation}
  \mathcal{L}_{int}\rightarrow \mathcal{L}_{int}+q(\partial_{\mu}\theta)\bar{\psi}\gamma^{\mu}\psi,
\end{equation}
which cancels out the non-gauge invariance seen in \EquationRef{eq:gaugeviolating}.

Ignoring the free term of the new gauge field the Lagrangian is now:
\begin{equation}
  \mathcal{L}=i\bar{\psi}\slashed{\partial}\psi-m\bar{\psi}\psi+q(\bar{\psi}\gamma^{\mu}\psi) A_{\mu}.
\end{equation}
This can be rewritten as:
\begin{equation}
  \label{eq:localdiraclagrangian}
  \mathcal{L}=i\bar{\psi}\gamma^{\mu}\mathcal{D}_{\mu}\psi-m\bar{\psi}\psi,
\end{equation}
where $\mathcal{D}_{\mu}=\partial_{\mu}+iqA_{\mu}$ and is referred to as the covariant derivative. Comparing \EquationRef{eq:globaldiraclagrangian} and \EquationRef{eq:localdiraclagrangian} it can be seen that in order to go from a globally invariant Lagrangian to a locally invariant one we have substituted the normal spacetime derivative for the covariant derivative.

$U(1)$ transformations have one degree of freedom and can be described by one parameter, in the above case $\theta$, and in order to make the Lagrangian locally invariant one interacting gauge boson had to be added. This correspondence between the number of degrees of freedom and the number of gauge bosons holds generally. For each degree of freedom of a group's transformations there exists a generator of the group, and for each generator one interacting gauge boson must be added to achieve local invariance.

%!!give SM gauge group and vector representations maybe move first bit of next section here

\subsection{Mass and spontaneous symmetry breaking}
%!!Introduce SSB
%!!Describe electroweak symmetry breaking in the Standard Model
In \EquationRef{eq:localdiraclagrangian} we ignored the free term for the gauge field. The free term for a general vector field is:

\begin{equation}
  \label{eq:vectorlagrangian}
  \mathcal{L}_{V}=-\frac{1}{4}F_{\mu\nu}F^{\mu\nu}+m_{V}A_{\mu}A^{\mu},
\end{equation}
where $F_{\mu\nu}$ is the field strength tensor of the field and $m_{V}$ is its mass. While the first term in this Lagrangian is gauge invariant, the second term is not. Therefore, massive vector bosons are not possible on their own in gauge invariant theories.

In the SM fermion mass terms are also not possible as 
%!!ALSO CONSIDER FERMION MASS, NOT GAUGE INVARIANT

%!!SOLUTION SSB


\subsection{Gauge symmetries in the SM}
%!!Introduce SM gauge group
%!!Describe representation of fermions
%!!Covariant derivative of SM, how it gives forces, briefly
%!!Talk about mass term and Higgs
The SM is gauge invariant under the group $SU\left(3\right)_{C}\otimes SU\left(2\right)_{L}\otimes U\left(1\right)_{Y}$. 

Fermions in the SM are spin half representations of the above symmetry groups.

The vector bosons in the standard model emerge as a result of the gauge invariance described above 






\subsection{Higgs boson production and decay at the LHC}
%!!Detail Higgs production and decays in order to motivate searching for invisible decays in the VBF channel

%!!DISCUSS HIGGS

\subsection{Challenges for the SM}
\label{sec:SMchallenges}


\section{Dark matter}
\label{sec:DM}
%!!Introduce dark matter and motivate search for it and other invisible final states in Higgs decays

\section{Some extensions of the standard model incorporating dark matter}
\label{sec:DMextensions}
%!!Introduce theories that are used for pheno work
